\documentclass[12pt,letterpaper]{article}           % fleqn: align equations left

% Document:
\usepackage{geometry}                                     % Custom margins for single page, etc.
\usepackage{fullpage}                                     % Use the full page
\usepackage{setspace}                                     % Enables custom margins, doublespacing, etc.
\usepackage{pdflscape}                                    % Use: \begin{landscape} ... \end{landscape}

% Font/text:
\usepackage[latin9]{inputenc}                             % Font definition and input type
\usepackage[T1]{fontenc}                                  % Font output type
\usepackage{lmodern}                                      % Latin Modern fonts
\usepackage{textcomp}                                     % Supports many additional symbols

%\usepackage[urw-garamond]{mathdesign} %dont use with amssymb, etc
%%%%Packages used without Mathdesign
\usepackage{amsmath}                                      % Math equations, etc.
\usepackage{amsthm}                                       % Math theorems, etc.
\usepackage{amsfonts}                                     % Math fonts (e.g. script fonts)
\usepackage{amssymb}                                      % Math symbols such as infinity
\DeclareMathOperator*{\Max}{Max}                          % Better looking max function
\DeclareMathOperator*{\Min}{Min}                          % Better looking min function

%%%%%%%%%%%%%%%%%%%%%%%5
\usepackage{xcolor}                                        % Enables colored text
\definecolor{darkblue}{rgb}{0.0,0.0,0.66}                 % Custom color: dark blue

% Images:
\usepackage{graphicx}                                     % Allows .jpg images to be included
%\usepackage{epstopdf}                                    % Convert .eps images on the fly
%\usepackage{subfig}                                       % Enables arrayed images
\usepackage[section]{placeins}                            % Forces floats to stay in section
\usepackage{float}                                        % Used with restylefloat
\restylefloat{figure}                                     % "H" forces a figure to be "exactly here"
\usepackage[justification=centering]{caption}             % Center captions
\usepackage{subfig} %allow multiple floats in a figure



%\usepackage{datetime}                                     % Custom date format for date field
%\newdateformat{mydate}{\monthname[\THEMONTH] \THEYEAR}    % Defining month year date format
\usepackage{tikz}                                        % Timelines and other drawings
\usetikzlibrary{decorations}                             % Formating for Tikz
\usetikzlibrary{calc}
\usetikzlibrary{matrix}
\usetikzlibrary{positioning}
\definecolor{darkred}{rgb}{0.8,0,0}
\usepackage{tikz-3dplot}


%%%%%%%%%%%%%%%%For Filling in the area between to curves
\usepackage{pgfplots}
\pgfplotsset{compat=1.11}
\usepgfplotslibrary{fillbetween}
\usetikzlibrary{intersections}
\usetikzlibrary{patterns}
\usepgfplotslibrary{ternary}


\pgfdeclarelayer{bg}
\pgfsetlayers{bg,main}

%%%%%%%%%%%MATH
\usepackage{graphicx}
\usepackage{caption}
%\usepackage{subcaption} %this produces an error with the package subfig

%Flow Chart
\tikzstyle{decision} = [diamond, draw, fill=blue!20, text width=4.5em, text badly centered, node distance=3cm, inner sep=0pt]
\tikzstyle{block}    = [rectangle, draw, fill=black!25, text width=5em, text centered, rounded corners, minimum height=4em]
\tikzstyle{line}     = [draw, -latex']
\tikzstyle{cloud}    = [draw, ellipse,fill=red!20, node distance=3cm, minimum height=2em]

\pagestyle{empty}

\usepackage{tikz}
\usepackage{calc} % for simple arithmetic
\tikzset{>=latex} % for LaTeX arrow head

% split figures into pages
\usepackage[active,tightpage]{preview}
\PreviewEnvironment{tikzpicture}
\setlength\PreviewBorder{1pt}%

%%%%%%%%%%%%%%%%%%%%%%%%%%%%%%%5
\begin{document}

\begin{figure}
% TIMELINE - simple
\begin{tikzpicture}[]

% limits
\newcount\yearOne; \yearOne=1900
\def\w{15}    % width of axes
\def\n{4}     % number of decades
\def\lt{0.40} %  ten tick length
\def\lf{0.36} % five tick length
\def\lo{0.30} %  one tick length

% help functions
\def\yearLabel(#1,#2){\node[above] at ({(#1-\yearOne)*\w/\n/10},\lt) {#2};}
\def\yearArrowLabel(#1,#2,#3,#4){
	\def\xy{{(#1-\yearOne)*\w/\n/10}}; \pgfmathparse{int(#2*100)};
	\ifnum \pgfmathresult<0
	\def\yyp{{(\lt*(0.90+#2))}}; \def\yyw{{(\yyp-\lt*#3)}}
	\draw[<-,thick,black,align=center] (\xy,\yyp) -- (\xy,\yyw) node[below,black] at (\xy,\yyw) {#4};
	\else
	\def\yyp{{(\lt*(0.10+#2)}}; \def\yyw{{(\yyp+\lt*#3)}}
	\draw[<-,thick,black,align=center] (\xy,\yyp) -- (\xy,\yyw) node[above,black] at (\xy,\yyw) {#4};
	\fi}

% axis
%\draw[thick] (0,0) -- (\w,0);
\draw[->,thick] (-\w*0.03,0) -- (\w*1.03,0);

% ticks
\foreach \tick in {0,1,...,\n}{
	\def\x{{\tick*\w/\n}}
	\def\year{\the\numexpr \yearOne+\tick*10 \relax}
	\draw[thick] (\x,\lt) -- (\x,-\lt) % ten tick
	node[below] {\year};
	
	\ifnum \tick<\n
	\draw[thick] ({(\x+\w/\n/2)},0) -- ({(\x+\w/\n/2)},\lf); % five tick
	\foreach \ticko in {1,2,3,4,6,7,8,9}{
		\def\xo{{(\x+\ticko*\w/\n/10)}}
		\draw[thick] (\xo,0) -- (\xo,\lo);  % one tick
	}\fi
}

% label
\yearLabel(1923,lol)
\yearArrowLabel(1932.2, 1.0,1.0,foo)
\yearArrowLabel(1937.2, 1.0,10.5,foo bar)
\yearArrowLabel(1907.5, 0.0,1.5,small)


\draw[->,thick] (-\w*0.03,4) -- (\w*1.03,7);
\draw [decorate,decoration={brace,amplitude=10pt,mirror,raise=4pt},yshift=0pt]
(5,0) -- (0,5) node [black,midway,xshift=1cm, yshift=.75cm] {$\frac{1}{2}- p(s)$};
\end{tikzpicture}
\end{figure}

\end{document}
